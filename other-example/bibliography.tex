\begin{thebibliography}{99}
\bibitem{1} Mädler S, Sun F, Tat C, Sudakova N, Drouin P, Tooley R.
Trace-Level Analysis of Hexavalent Chromium in Lake Sediment Samples
Using Ion Chromatography Tandem Mass Spectrometry, \emph{Journal of
Environmental Protection, 7}: 422--434, 2016.

doi: \href{https://doi.org/10.4236/jep.2016.73037}{10.4236/jep.2016.73037}

\bibitem{2} Xie Y, Holmgren S, Andrews D, Wolfe M. Evaluating the impact of
the US National toxicology program: A case study on hexavalent chromium,
\emph{Environmental health perspectives, 125}: 181--188, 2017.

doi: \href{http://dx.doi.org/10.1289/EHP21}{10.1289/EHP21}

\bibitem{3} Pineda M, Rodríguez A. Metales pesados (Cd, Cr y Hg): su impacto
en el ambiente y posibles estrategias biotecnológicas para su
remediación, \emph{Revista I3+, Investigación, Innovación, Ingeniería},
\emph{2}(2): 82--112, 2015.

doi: \href{https://doi.org/10.24267/23462329.113}{10.24267/23462329.113}

\bibitem{4} Williams P, Botes E, Maleke M, Ojo A, DeFlaun M, Howell J.
Effective bioreduction of hexavalent chromium--contaminated water in
fixed-film bioreactors, \emph{Water SA, 40}(3): 549--554, 2014.

doi: \href{http://dx.doi.org/10.4314/wsa.v40i3.19}{10.4314/wsa.v40i3.19}

\bibitem{5} Oves M., Khan M, Zaidi A. Chromium reducing and plant growth
promoting novel strain Pseudomonas aeruginosa OSG41 enhance chickpea
growth in chromium amended soils, \emph{European journal of soil
biology}, 56: 72--83, 2013.

doi: \href{https://doi.org/10.1016/j.ejsobi.2013.02.002}{10.1016/j.ejsobi.2013.02.002}

\bibitem{6} Hossan S, Hossain S, Islam MR, Kabir MH, Ali S, Islam MS, Mahmud
ZH. Bioremediation of Hexavalent Chromium by Chromium Resistant Bacteria
Reduces Phytotoxicity, \emph{Revista internacional de investigación
ambiental y salud pública}, 17: 6013, 2020.

doi: \href{https://doi.org/10.3390/ijerph17176013}{10.3390/ijerph17176013}

\bibitem{7} Mishra S, Chen S, Saratale G, Saratale R, Ferreira L, Bilal M.,
Bharagava. Reduction of hexavalent chromium by Microbacterium
paraoxydans isolated from tannery wastewater and characterization of its
reduced products, \emph{Journal of Water Process Engineering,} 39:
101748, 2021.

doi: \href{https://doi.org/10.1016/j.jwpe.2020.101748}{10.1016/j.jwpe.2020.101748}

\bibitem{8} Karthik C, Elangovan N, Kumar TS, Govindharaju S, Barathi S,
Oves M, Arulselvi PI. Characterization of multifarious plant growth
promoting traits of rhizobacterial strain AR6 under Chromium (VI)
stress, \emph{Microbiological Research}, 204: 65--71, 2017.

doi: \href{https://doi.org/10.1016/j.micres.2017.07.008}{10.1016/j.micres.2017.07.008}

\bibitem{9} Arango C, Alzate M.
\href{http://www.icesi.edu.co/blogs/gestionintegralindustrial/files/2011/10/SIRAC-Curtiembres.pdf}{\emph{Proyecto gestión ambiental en la industria de curtiembre en Colombia,}}
Pasto, ~\emph{Bogotá DC: Centro Nacional de Producción mas Limpia-Sistema de Referenciación Ambiental (SIRAC) para el Sector Curtiembre en Colombia}. 2004.

\bibitem{10} Guerrero-Ceballos DL, Pinta-Melo J, Fernández-Izquierdo P,
Ibargüen-Mondragón E, Hidalgo-Bonilla SP, Burbano-Rosero EM. Eficiencia
en la reducción de Cromo por una bacteria silvestre en un tratamiento
tipo Batch utilizando como sustrato agua residual del municipio de
Pasto, Colombia,~\emph{Universidad y Salud},~\emph{19}(1): 102--115, 2017.

doi: \href{http://dx.doi.org/10.22267/rus.171901.74}{10.22267/rus.171901.74}

\bibitem{11} Fernández M, Le Borgne S. Electroforesis en gradiente
denaturante. En: Cornejo A, Serrato B, Rendón, Rocha MG,
\emph{Herramientas moleculares aplicadas en ecología: aspectos teóricos
y prácticos}, México, D.F, Primera edición: 14--17, 2014.

\bibitem{12} American Public Health, Association, American Water Works,
Federation, Water Environment. Standards Methods for the examination of
water and wastewater, Chromiun 117A Hexavalente chromiun, In Health AP,
Association AWW, Federation WE, 1999.

\bibitem{13} Burbano-Rosero M, Caetano de Almeida B, Otero-Ramírez I.
\href{https://www.researchgate.net/profile/Edith_Burbano-Rosero/publication/326905291_Manual_de_Biologia_Molecular-Procedimientos_Basicos/links/5b7de35292851c1e12291a22/Manual-de-Biologia-Molecular-Procedimientos-Basicos.pdf}{Manual de Biología Molecular -- Procedimientos Básicos}, \emph{Manual de Biología Molecular -- Procedimientos Básicos}, Pasto, Colombia, 2017: 13--50, 2017.

\bibitem{14} Sambrook J, Fritschi EF, Maniatis T. Molecular cloning: a
laboratorymanual, Cold Spring Harbor Laboratory Press, New York, 1989.

\bibitem{15} Brosius J, Dull TJ, Sleeter D, Noller H. Gene organization and
primary structure of ribosomal RNA operon from \emph{Escherichia coli},
\emph{Journal of molecular biology,} 148: 107--127, 198.

doi: \href{https://doi.org/10.1016/0022-2836(81)90508-8}{10.1016/0022-2836(81)90508-8}

\bibitem{16} Robalino S, Wilson C. 
\href{https://dspace.ups.edu.ec/handle/123456789/14669}{Identificación molecular del complejo Burkholderia cepacia, bacteria productora de antibióticos, mediante PCR en tiempo real}, 
\emph{Disertación Tesis}, Universidad Politécnica
Salesiana, Quito. 2017.

\bibitem{17} Genovese M, Crisafi F, Denaro R, Cappell S, Russo D, Calogero
R, Genovese L. Effective bioremediation strategy for rapid in situ clean
up of anoxic marine sediments in mesocosm oil spill simulation,
\emph{Frontiers in microbiology, 5}(162): 1--14, 2014.

doi: \href{https://doi.org/10.3389/fmicb.2014.00162}{10.3389/fmicb.2014.00162}

\bibitem{18} Barton LL, Northup DE. Microbes at work in nature:
biomineralization and microbial weathering,~\emph{Microbial Ecology},
2011: 299--326, 2011.

\bibitem{19} Faissal , Ouazzani N, Parrado J, Dary M, Manyani H, Morgado B.
Impact of fertilization by natural manure on the microbial quality of
soil: Molecular Approach, \emph{Saudi journal of biological sciences,
24}(6): 1437--144, 2017.

doi: \href{https://doi.org/10.1016/j.sjbs.2017.01.005}{10.1016/j.sjbs.2017.01.005}

\bibitem{20} Shahsavari E, Aburto-Medina A, Khudur LS, Taha M, Ball AS.
Microbial Ecology to Microbial Ecotoxicology. En Cravo-Laureau C, Cagnon
C, Lauga B, Duran R, In~\emph{Microbial Ecotoxicology}, New York:
Springer, 17--38, 2017.

doi: \href{https://doi.org/10.1007/978-3-319-61795-4}{10.1007/978-3-319-61795-4}

\bibitem{21} Fernández M, Le-Borgne S. Electroforesis en gel con gradiente
desnaturalizante, En: Cornejo R, Serrato D, Aguilar B, Munive M,
\emph{Herramientas moleculares aplicadas en ecología: Aspectos teóricos
y prácticos}, SEMARNT INEC UAM-I, 149--170, 2014.

\bibitem{22} Neilson J, Jordan F, Maier R. Analysis of artifacts suggests
DGGE should not be used for quantitative diversity analysis,
\emph{Journal of Microbiological Methods, 92}(2013): 256--263, 2013.

doi: \href{https://doi.org/10.1016/j.mimet.2012.12.021}{10.1016/j.mimet.2012.12.021}

\bibitem{23} Banerjee S, Misra A, Chaudhury S, Dam B. A Bacillus strain TCL
isolated from Jharia coalmine with remarkable stress responses, chromium
reduction capability and bioremediation potential, Journal of hazardous
materials, 367: 215--223, 2019.

doi: \href{https://doi.org/10.1016/j.jhazmat.2018.12.038}{10.1016/j.jhazmat.2018.12.038}

\bibitem{24} Buckhout-White S, Person C, Medintz IL, Goldman ER. Restriction
Enzymes as a Target for DNA-Based Sensing and Structural Rearrangement,
\emph{ACS Omega}, 3(1): 495--502, 2018.

doi: \href{https://doi.org/10.1021/acsomega.7b01333}{10.1021/acsomega.7b01333}

\bibitem{25} Di Felice F, Micheli G, Camilloni G. Restriction enzymes and
their use in molecular biology: An overview, \emph{Journal of
biosciences},~\emph{44}(2): 38, 2019.

doi: \href{https://doi.org/10.1007/s12038-019-9856-8}{10.1007/s12038-019-9856-8}

\bibitem{26} Cruz-Leyva M, Zamudio-Maya M, Corona-Cruz A, González- de la
Cruz J, Rojas-Herrera R.
\href{http://www.scielo.org.mx/scielo.php?pid=S2007-90282015000100008\&script=sci_arttext}{Importancia y estudios de las comunidades microbianas en los recursos y productos pesqueros}, \emph{Ecosistemas y
recursos agropecuarios}, 2(4): 99--115, 2015.

\bibitem{27} Chai L, Yang Z, Shi Y, Liao Q, Min X, Li Q, Liang L. Cr
(VI)-reducing strain and its application to the microbial remediation of
Cr (VI)-contaminated soils, \emph{In Twenty years of research and
development on soil pollution and remediation in China}, 2018: 487--498,
2018.

doi: \href{https://doi.org/10.1007/978-981-10-6029-8_29}{10.1007/978-981-10-6029-8\_29}

\bibitem{28} Ma L, Xu J, Chen N, Li M, Feng C. Microbial reduction fate of
chromium (Cr) in aqueous solution by mixed bacterial consortium,
\emph{Ecotoxicology and environmental safety}, 170: 763--770, 2019.

doi: \href{https://doi.org/10.1016/j.ecoenv.2018.12.041}{10.1016/j.ecoenv.2018.12.041}

\bibitem{29} Lin H, You S, Liu L. Characterization of Microbial Communities,
Identification of Cr(VI) Reducing Bacteria in Constructed Wetland and
Cr(VI) Removal Ability of Bacillus cereus, \emph{Scientific Reports}, 9:
12873, 2019.

doi: \href{https://doi.org/10.1038/s41598-019-49333-4}{10.1038/s41598-019-49333-4}

\bibitem{30} Yin P, Liu X, Liao J, Hu X. Effects of Cadmium Stress on
Microbial Community Diversity in Soil Potted With Sasa Argenteastriatus,
\emph{In IOP Conference Series: Earth and Environmental Science}, 300:
052051, 2019.

doi: \href{https://doi.org/10.1088/1755-1315/300/5/052051}{10.1088/1755-1315/300/5/052051}
        
\end{thebibliography}
